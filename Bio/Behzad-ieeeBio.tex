%-------------------------------------------------------------------------------
% Behzad NOURI's IEEE Biography:
%-------------------------------------------------------------------------------
\begin{IEEEbiography}[{\includegraphics[width=1.08in,height=1.6in,clip,keepaspectratio]{Photo/behzadnouri2}}]{Behzad~Nouri~(S'08-M'14)} received the B. Eng. degree in Electronics Engineering from Tabriz University, Iran, and M.Sc. degree in Electrical Engineering and Ph.D. degree in Electrical and Computer Engineering from Carleton University, Ottawa, Canada in 2008 and 2014, respectively. \par 
 
Dr. Nouri currently is a postdoctoral research fellow and Research Adjunct Professor with the department of electronics engineering at Carleton University. Prior to attending in Carleton university (2005), he served in various capacities in leading research labs and telecommunication industries, including Iran Telecommunication Research Center (ITRC, 1989), Iran Communication Industries Inc. (ICII, 1990-1999), and Taher Telecom (1999-2002). His research interests include computer-aided design of VLSI circuits, signal/power integrity analysis, circuit simulation, model order reduction techniques for linear/nonlinear dynamical systems, differential geometry, parallel and numerical algorithms.  \par

Dr. Nouri received university gold medal for the outstanding doctoral work (2014). He was the recipient of the best paper award presented at the \mbox{15th} IEEE Workshop on Signal Propagation on Interconnects (2011) and a co-recipient of the IEEE Component, Packaging and Manufacturing Technology (CPMT) best transactions paper award (2013). He was also the recipient of prestigious scholarships during his graduate studies, including: Ontario graduate scholarship Award (2010), dean of graduate studies academic excellence scholarship (2008-2011), graduate departmental excellence scholarship (2008-2012)\@. \par

His  contribution in  perpetuating a positive learning  environment, and in implementing innovative teaching practices in electronics department was recognized with the departmental teaching assistance excellence awards (2011, 2012) and a nomination (by the department of electronics) for the university outstanding teaching assistant award (2013).
\end{IEEEbiography}